\begin{titlepage}

\begin{tabular}{c c}

\small ĐẠI HỌC QUỐC GIA TP.HCM & \small \textbf{CỘNG HÒA XÃ HỘI CHỦ NGHĨA VIỆT NAM}\\
\textbf{TRƯỜNG ĐẠI HỌC BÁCH KHOA}& \textbf{Độc lập - Tự do - Hạnh phúc}\\
--------------------------& ---------------------------
\end{tabular}

\vspace{1cm}
\centerline{\Large \textbf{NHIỆM VỤ LUẬN VĂN THẠC SĨ}}
\vspace{1cm}
\begin{tabular}{p{7cm} p{4cm}}
Họ tên học viên: \textbf{Nguyễn Văn Dương} & MSHV: \textbf{7140225} \\
Ngày, tháng, năm sinh: \textbf{06/11/1988} & Nơi sinh: \textbf{Vĩnh Long} \\
Chuyên ngành: Khoa học máy tính & Mã số: 60.48.01\\
\end{tabular}\\\\
\textbf{I. TÊN ĐỀ TÀI:}

So sánh hai phương pháp thu gọn tập huấn luyện RHC và Naive Ranking trong phân lớp dữ liệu chuỗi thời gian\\
\hspace{1cm} \textbf{II. NHIỆM VỤ VÀ NỘI DUNG}

\textbf{Nhiệm vụ}: Đề xuất giải thuật học chuyển đổi để hỗ trợ gom cụm gia tăng trong lĩnh vực giáo dục.

\textbf{Nội dung}:
\begin{enumerate}
\item Nghiên cứu lý thuyết về giải thuật phân lớp k-NN và các kỹ thuật thu gọn tập huấn luyện Naive Ranking, RHC, dRHC.
\item Hiện thực giải thuật phân lớp k-NN và các kỹ thuật thu gọn tập huấn luyện Naive Ranking, RHC, dRHC.
\item Thực nghiệm và so sánh kết quả phân lớp k-NN thuần tuý và phân lớp có kết hợp các kỹ thuật thu gọn tập huấn luyện Naive Ranking, RHC, dRHC.

\end{enumerate}
\textbf{III. NGÀY GIAO NHIỆM VỤ:} 26/02/2018 \\
\textbf{IV. NGÀY HOÀN THÀNH NHIỆM VỤ:} 17/06/2018 \\
\textbf{V. CÁN BỘ HƯỚNG DẪN:} PGS. TS. Dương Tuấn Anh \\\\
\begin{tabular}{c p{2cm} c}
 & &Tp.HCM, ngày ... tháng ... năm 2018\\
\textbf{CÁN BỘ HƯỚNG DẪN}& & \textbf{CHỦ NHIỆM BỘ MÔN ĐÀO TẠO}\\
\tiny (Họ tên và chữ ký) & & \tiny (Họ tên và chữ ký)\\

\end{tabular}

\vspace{0.5cm}
\centerline {\textbf{TRƯỞNG KHOA}}
\centerline {\tiny (Họ tên và chữ ký)}

\end{titlepage}

\newpage