\documentclass[11pt,a4paper,twoside,openany]{book}


\usepackage[latin1]{inputenc}
\usepackage[english]{babel}
\usepackage[demo]{graphicx}

% privides the H option
\usepackage{float}

% produces random text for testing
\usepackage{blindtext}

% Parameters for floating objects in LaTeX
% An overview can be found in the book
% The Latex Companions Chapter 6.1
% A good start is
% http://robjhyndman.com/researchtips/latex-floats/

\setcounter{topnumber}{2}
\setcounter{bottomnumber}{2}
\setcounter{totalnumber}{4}
\renewcommand{\topfraction}{0.85}
\renewcommand{\bottomfraction}{0.85}
\renewcommand{\textfraction}{0.15}
\renewcommand{\floatpagefraction}{0.8}
\renewcommand{\textfraction}{0.1}
\setlength{\floatsep}{5pt plus 2pt minus 2pt}
\setlength{\textfloatsep}{5pt plus 2pt minus 2pt}
\setlength{\intextsep}{5pt plus 2pt minus 2pt}

\begin{document}

\chapter{Test}

\blindtext[2]

\begin{figure}
\centering
\includegraphics[width=0.8\textwidth,height=50mm]{}
\caption{Test Test Test}
\end{figure}

\section{Test}

\blindtext[2]

\begin{figure}
\centering
\includegraphics[width=0.8\textwidth,height=50mm]{}
\caption{Test Test Test}
\end{figure}

\blindtext[2]

\begin{figure}
\centering
\includegraphics[width=0.8\textwidth,height=50mm]{}
\caption{Test Test Test}
\end{figure}

\section{Test}

\blindtext[2]

\begin{figure}
\centering
\includegraphics[width=0.8\textwidth,height=50mm]{}
\caption{Test Test Test}
\end{figure}

\blindtext[2]

\end{document}